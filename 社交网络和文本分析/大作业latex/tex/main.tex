%%%%%%%%%%%%%%%%%%%%%%%%%%%%%%%%%%%%%%%%%%%%%%%%%%%%%%%%%%%%%%%%%%
%%%%%%%%%%%%%%%%%%%%%%%%%%%%%%%%%%%%%%%%%%%%%%%%%%%%%%%%%%%%%%%%%%
%Packages
\documentclass[10pt, a4paper]{article}
\usepackage{naturetex}
\usepackage[normalem]{ulem}
\usepackage[top=3cm, bottom=4cm, left=3.5cm, right=3.5cm]{geometry}
\usepackage{amsmath,amsthm,amsfonts,amssymb,amscd, fancyhdr, color, comment, graphicx, environ}
\usepackage{float}
\usepackage{mathrsfs}
\usepackage[math-style=ISO]{unicode-math}
\setmathfont{TeX Gyre Termes Math}
\usepackage{lastpage}
\usepackage[dvipsnames]{xcolor}
\usepackage[framemethod=TikZ]{mdframed}
\usepackage{enumerate}
\usepackage[shortlabels]{enumitem}
\usepackage{ctex}
\usepackage{fancyhdr}
\usepackage{indentfirst}
\usepackage{listings}
\usepackage{sectsty}
\usepackage{thmtools}
\usepackage{shadethm}
\usepackage{hyperref}
\usepackage{setspace}
\hypersetup{
	colorlinks=true,
	linkcolor=blue,
	filecolor=magenta,      
	urlcolor=blue,
}
%%%%%%%%%%%%%%%%%%%%%%%%%%%%%%%%%%%%%%%%%%%%%%%%%%%%%%%%%%%%%%%%%%
%%%%%%%%%%%%%%%%%%%%%%%%%%%%%%%%%%%%%%%%%%%%%%%%%%%%%%%%%%%%%%%%%%
%Environment setup
\mdfsetup{skipabove=\topskip,skipbelow=\topskip}
\newrobustcmd\ExampleText{%
	An \textit{inhomogeneous linear} differential equation has the form
	\begin{align}
		L[v ] = f,
	\end{align}
	where $L$ is a linear differential operator, $v$ is the dependent
	variable, and $f$ is a given non−zero function of the independent
	variables alone.
}

\mdtheorem[style=theoremstyle]{Problem}{Problem}
\newenvironment{Solution}{\textbf{Solution.}}

%%%%%%%%%%%%%%%%%%%%%%%%%%%%%%%%%%%%%%%%%%%%%%%%%%%%%%%%%%%%%%%%%%
%%%%%%%%%%%%%%%%%%%%%%%%%%%%%%%%%%%%%%%%%%%%%%%%%%%%%%%%%%%%%%%%%%
%Fill in the appropriate information below
\newcommand{\norm}[1]{\left\lVert#1\right\rVert}     
\newcommand\course{Course}                      % <-- course name   
\newcommand\hwnumber{1}                         % <-- homework number
\newcommand\Information{XXX/xxxxxxxx}           % <-- personal information
%%%%%%%%%%%%%%%%%%%%%%%%%%%%%%%%%%%%%%%%%%%%%%%%%%%%%%%%%%%%%%%%%%
%%%%%%%%%%%%%%%%%%%%%%%%%%%%%%%%%%%%%%%%%%%%%%%%%%%%%%%%%%%%%%%%%%
%Page setup
\pagestyle{fancy}
\headheight 40pt
\lhead{\today}
\rhead{\includegraphics[width=2.5cm]{西安交通大学logo2}} % <-- school logo(please upload the file first, then change the name here)
\lfoot{}
\pagenumbering{arabic}
\cfoot{\small\thepage}
\rfoot{}
\headsep 1.25em
\renewcommand{\baselinestretch}{1.25}       
\mdfdefinestyle{theoremstyle}{%
	linecolor=black,linewidth=1pt,%
	frametitlerule=true,%
	frametitlebackgroundcolor=gray!20,
	innertopmargin=\topskip,
}
%%%%%%%%%%%%%%%%%%%%%%%%%%%%%%%%%%%%%%%%%%%%%%%%%%%%%%%%%%%%%%%%%%
%%%%%%%%%%%%%%%%%%%%%%%%%%%%%%%%%%%%%%%%%%%%%%%%%%%%%%%%%%%%%%%%%%
%Add new commands here
\renewcommand{\labelenumi}{\alph{enumi})}
\newcommand{\Z}{\mathbb Z}
\newcommand{\R}{\mathbb R}
\newcommand{\Q}{\mathbb Q}
\newcommand{\NN}{\mathbb N}
\DeclareMathOperator{\Mod}{Mod} 
\renewcommand\lstlistingname{Algorithm}
\renewcommand\lstlistlistingname{Algorithms}
\def\lstlistingautorefname{Alg.}
%%%%%%%%%%%%%%%%%%%%%%%%%%%%%%%%%%%%%%%%%%%%%%%%%%%%%%%%%%%%%%%%%%
%%%%%%%%%%%%%%%%%%%%%%%%%%%%%%%%%%%%%%%%%%%%%%%%%%%%%%%%%%%%%%%%%%
%Begin now!



\begin{document}
	
	\begin{titlepage}
		\begin{center}
			\vspace*{3cm}
			
			\Huge
			\textbf{《机器学习2》课程报告8-9讲}
			
			\vspace{1cm}
	
			\vspace{3.5cm}
					\includegraphics[width=0.7\textwidth]{西安交通大学logo2}
			
			\vfill
			
			
			\Large
			\begin{center}
				\begin{tabular}{rl}
					% after \\: \hline or \cline{col1-col2} \cline{col3-col4} ...
					\textbf{小组成员:}& 赵敬业、苏煜家、杨鸿谦、熊子骁 
				\end{tabular}
			\end{center}




			
			\Large
			
			\today
			
		\end{center}
	\end{titlepage}
	
	%%%%%%%%%%%%%%%%%%%%%%%%%%%%%%%%%%%%%%%%%%%%%%%%%%%%%%%%%%%%%%%%%%
	%%%%%%%%%%%%%%%%%%%%%%%%%%%%%%%%%%%%%%%%%%%%%%%%%%%%%%%%%%%%%%%%%%
	%Start the assignment now
	
	%%%%%%%%%%%%%%%%%%%%%%%%%%%%%%%%%%%%%%%%%%%%%%%%%%%%%%%%%%%%%%%%%%
	%New problem
% 目录
\newpage
\tableofcontents
\newpage

\section{第八讲 字典学习}
%%figure在figure文件夹中可以看到,只需要写出图片名字就行
\subsection{回归分析}

回归分析:是一种统计学上分析数据的方法,目的在于了解两个或多个变量间是否相关、相关方向与强度,并建立数学模型以便观察特定变量来预测研究者感兴趣的变量。

线性方法解的形式:$f(x)=w_0+w_1x^{(1)}+w_2x^{(2)}+\ddots+w_px^{(p)}$

线性方法优缺点:
\begin{itemize}
	\item 优点
	\begin{itemize}
		\item 易建模
		\item 易解释
		\item 易计算
	\end{itemize}
	\item 缺点
	\begin{itemize}
		\item 精度一般
		\item 抗干扰能力差
		\item 先验依赖强
	\end{itemize}
\end{itemize}

线性方法的三要素:
\begin{itemize}
	\item 假设空间:线性关系假设。也是线性模型的一个关键问题。这里可以嵌入线性模型的很多先验知识。比如音乐数据MIT团队嵌入了一些变量重要、一些不重要的先验信息取得了较好的结果;但是实际上很多自变量对因变量的影响是非线性的,如销量和价格的关系。
	\item 优化策略: 最小化残差
	\item 学习算法: 这多了去了:OLS方法、所有能想到的算法
\end{itemize}

\subsection{广义可加模型(GAM)}

公式:
$$
f(x)=\mathrm{w}_{0}+\mathrm{w}_{1} g_{1}\left(x^{(1)}\right)+\mathrm{w}_{2} g_{2}\left(x^{(2)}\right)+\ldots+w_{p} g_{p}\left(x^{(p)}\right)
$$
其中,用g刻画各个变量的影响。

至于实施步骤,广义可加模型具有以下步骤
\begin{enumerate}
	\item 选择假设空间。假设空间可以选择任何函数,但是一般选择光滑函数,并且可以由此嵌入先验信息,比如
	\begin{itemize}
		\item 样条基函数
		\item 回归树
		\item 多项式
	\end{itemize}
	\item 选择优化策略。优化策略选择范围比较广泛。如
	\begin{itemize}
		\item OLS
		\item RLS
		\item LASSO
		\item 其他正则化模型
	\end{itemize}
	\item 选择算法。可以选择包括
	\begin{itemize}
		\item 解析表达
		\item 梯度下降
	\end{itemize}
\end{enumerate}

自变量无关性假设

\subsection{字典学习}

核心思想:无论人类的知识有多么浩繁,也无论人类的科技有多么发达,一本新华字
典或牛津字典足以表达人类从古至今乃至未来的所有知识,那些知识只不
过是字典中字的排列组合罢了!

广义可加模型只是对一个变量进行了操作。而字典学习是对所有的变量进行某种组合,如公式
$$
f(x)=\mathrm{w}_{1} h_{1}(x)+\mathrm{w}_{2} h_{2}(x)+\ldots+w_{N} h_{N}(x)
$$

优缺点:
\begin{itemize}
	\item 优点。 字典学习实质上是对于庞大数据集的一种降维表示,或者说是信息的压缩。正如同字是句子最质朴的特征一样,字典学习总是尝试学习蕴藏在样本背后最质朴的特征。
	\item 缺点。解释性一般,计算量大,需要一定的数学基础
\end{itemize}

步骤
\begin{enumerate}
	\item 假设空间选择。选择一簇函数近似学习目标。可编码。
	\begin{enumerate}
		\item 基的选择:多项式;回归树;核函数;神经网络。
		\item N的选择:理论上越大越好,算法上适当大即可。
	\end{enumerate}
	\item 优化策略选择。可求解是优化策略必须的能力,可解释是应该争取的,解释性比较强的模型对管理者更为友好。
	\begin{enumerate}
		\item 目标函数。目标函数依然包括经验风险最小化。
		\item 约束条件。即为结构风险极小化,具体可以包括二范数约束、一范数约束、核范数约束等。
	\end{enumerate}
	\item 算法选择。求解所建立的数学模型。有效性、快速性。其中正则化参数的作用是使方程有惟一解,控制解的结构、提高泛化能力等。
	\begin{enumerate}
		\item 二范数。可以使用解析表达式(其实也可以用下降方法,反正是凸的)。问题定义:$$
		\begin{aligned}
			&\operatorname{argmin} \frac{1}{m} \sum_{i=1}^{m}\left(f\left(x_{i}\right)-y_{i}\right)^{2} \\
			&\text { s.t. } \sum_{j=1}^{N}\left|w_{j}\right|^{2} \leq t
		\end{aligned}
		$$
		其中$$
		Y=\left(\begin{array}{c}
			y_{1} \\
			y_{2} \\
			\vdots \\
			y_{m}
		\end{array}\right) \quad H=\left(\begin{array}{ccccc}
			h_{1}\left(x_{1}\right) & h_{2}\left(x_{1}\right) & h_{3}\left(x_{1}\right) & \cdots & h_{N}\left(x_{1}\right) \\
			h_{1}\left(x_{2}\right) & h_{2}\left(x_{2}\right) & h_{3}\left(x_{2}\right) & \cdots & h_{N}\left(x_{2}\right) \\
			\vdots & \vdots & \vdots & & \vdots \\
			h_{1}\left(x_{m}\right) & h_{2}\left(x_{m}\right) & h_{3}\left(x_{m}\right) & \cdots & h_{N}\left(x_{m}\right)
		\end{array}\right) \quad W=\left(\begin{array}{c}
			\mathcal{W}_{0} \\
			\mathcal{W}_{1} \\
			\mathcal{W}_{2} \\
			\vdots \\
			\mathcal{W}_{N}
		\end{array}\right)
		$$
		解析解:$$W_{2, \lambda}=\left(\mathrm{H}^{\prime} H+\lambda I\right)^{-1} X^{\mathrm{T}} Y$$
		\item 一范数。可以使用梯度下降等方法。具体算法如下:
		$$\rho_j=\frac{1}{m}\sum_{i=1}^{m}h_j(x)(y_j-\sum_{k!=j}w_kh_k(x_i))$$
		$$\hat{w_j}=\left\{\begin{array}{cc}
			(\rho_j+\lambda/2)/z_j, & \rho_j>\lambda/2\\
			0, & \rho_j \in \left(-\lambda/2,\lambda/2\right) \\
			(\rho_j-\lambda/2)/z_j, & \rho_j<-\lambda/2
		\end{array}\right.$$
		\item 零范数。可以使用贪婪算法,计算复杂度$O((k^*)^3)$。步骤如下:
		\begin{enumerate}
			\item 初始化:$f_0(x)=0,r_0(x)=\sum_{i=1}^{m}(f_0(x_i)-y_i)^2$
			\item 贪婪选基$h_k^*=\arg\min|\sum_{i=1}^{m}h(x_i)r_{k-1}(x_i)|$
			\item 贪婪选权重$w_k^*=\arg\min\sum_{i=1}^{m}(w_1h_1^*(x_1)+\ddots+w_kh_k^*(x_i)-y_i)$
			\item 更新残差$f_k(x)=(w_1h_1^*(x_1)+\ddots+w_kh_k^*(x_i),r_0(x)=\sum_{i=1}^{m}(f_k(x_i)-y_i)^2$
			\item 更新k,回到第二步或停机
		\end{enumerate}
	\end{enumerate}
\end{enumerate}

其中需要说明的是稀疏性选择具有比较大的意义。比如对于一个新的知识,生疏者会有更多的神经元被激活,熟练者神经元会有更少被激活;雷达只有在稀疏时才能有意义。

\section{第九讲 神经网络}
\subsection{神经网络的原理}

核方法是以再生核希尔伯特空间为假设空间,以结构风险极小化作为优化策略,以梯度型方法作为学习算法的一种方法;字典学习以冗余字典作为假设空间。这两种方法都是在足够大(核方法是恰好足够大)的空间中进行学习,获得一个用来逼近数据的低维空间。

从假设空间上来看,神经网络的假设空间与前两种提到的方法不一样,神经网络的假设空间是无限维的,基底具体长什么样子由训练产生;并且神经网络非线性非凸,而前两种方法最后都是线性模型。

神经网络可以从仿生的角度进行解释。生物神经网络包括细胞体、树突、突触和轴突等部分,分别对应神经元、内权、阈值和外权。
$$\sum_{k=1}^{n}a_k\sigma(w_k x+\theta_k)$$

如上面公式所述,其中$a_k$代表外权;函数$\sigma$代表激活函数,模拟了生物的兴奋和抑制状态;$w_k$即为内权;$\theta$即为阈值,模拟了突触延时和不应期。

\subsection{神经网络的分类}

按照连接方式分,可以将神经网络分成前馈网络、反馈网络和图神经网络。前馈网络一般用来做预测,反馈网络一般用来做自然语言处理,图神经网络最近比较火热,但具体能做什么还不清楚,笔者了解到的有使用这种方法预测推特用户是否是水军的。

按照隐层数量分,可以将神经网络分为浅层和深度神经网络。这里将一层的神经网络定义为浅层,多层的定义为多层,\sout{以满足发表论文、对外忽悠人的需要},本节中只讨论浅层神经网络,作为对照,深度神经网络有很多优点以及不同之处,留在之后讨论。

按照隐层数和连接方式分类,可以分为全连接神经网络和卷积神经网络、梳妆神经网络等。不同的神经网络结构可以用来嵌入不同的先验知识。

\subsection{神经网络的三要素}
\subsubsection{假设空间}
$$H_n=\left\{f(x)=\sum_{k=1}^{n}a_k\sigma(w_k x+\theta_k)\right\}$$
在假设空间中,需要确定神经元个数和激活函数。其中激活函数主要有
\begin{itemize}
	\item Sigmoid函数。所有从0突变为1的光滑函数都是Sigmoid函数,著名的是逻辑函数$f(x)=\frac{1}{1+e^{-x}}$,这种激活函数主要问题是在求梯度的过程中容易出现问题,当函数值离远点较远,则梯度非常小,容易导致梯度消失,以及出现饱和问题。
	\item tanh函数。$tanh(x)=\frac{e^x-e^{-x}}{e^x+e^{-x}}$,这种函数和Sigmoid没有本质区别,只是函数值域变为(-1,1)
	\item Relu函数。$Relu=max(0,x)$克服了前面梯度消失的问题,因为当存在梯度的时候,梯度一定是1,但是不光滑导致求导困难,而且一旦进入了负数区域导致神经元被抑制就永远被抑制。
	\item ELU函数。有人尝试寻找能好处占尽的激活函数而创造了ELU, 正数部分和Relu一样,保证了梯度不消失,在非正数区域换成了$\alpha(e^{x}-1)$,尽量避免了梯度消失、神经元永不激活的问题,但因为计算量大导致应用较少\sout{和笔者建立的复杂的模型一样拉跨}。
\end{itemize}
\subsubsection{优化策略}

优化策略包括损失函数选择和罚函数选择。值得一提的是十年前有一位高人非常擅长调神经网络,就是巧妙利用了罚函数,笔者也听说有人试图用$Fisher$信息熵作为罚函数,不知道效果如何。

\subsubsection{学习算法}

神经网络假设空间和优化策略都是可圈可点的地方,而学习算法是硬伤(好像所有稍微复杂一点的、非凸非光滑的模型都这样????)。学习算法和优化策略不适配,是神经网络没有解释性的根本原因。学习算法选择一般选择梯度型的算法,比如误转逆传播算法。
\subsection{浅层神经网络的优缺点}
\subsubsection{优点}

维数无关\cite{barron1993universal}。神经网络的逼近性质和维数没有关系,可避免维数祸根问题。

线性方法能做的,神经网络一定能做。\cite{mhaskar1996neural}(我看不懂,老师让我引用的)。

\subsubsection{缺点}

饱和性问题\cite{debao1993degree}。浅层神经网络面临者饱和性的问题。

稀疏性问题\cite{chui1994neural}。浅层神经网络面临着稀疏性问题,即无法模拟稀疏问题。

旋转不变性。浅层神经网络在旋转不变性的模拟方面不能有很好的效率,导致在地震预测等方面效果不好。

在其他一些结构上,浅层神经网络表现也不好,如一些特殊的结构、流形问题等。

\subsection{训练方法}

神经网络的训练包括两个层面的参数,即显参数和隐参数。按照这种分类方法,可以同时训练显参数和因参数,即为单场景训练;分别训练显参数和因参数,即为双场景训练。

\subsubsection{双场景训练}

内权是较为典型的隐参数,这里记为T。可以通过ADMM算法进行训练得到,也可以通过随机赋值、还可以直接强行嵌入领域知识。

外权直接通过线性的方法进行训练即可。

\section{第八讲 勘误}

考虑到老师平时工作繁忙,本人(赵敬业)也因考研等原因没有时间亲自叩门拜访老师,特在此处将整理过程中PPT第八讲可能出现的错误罗列至此,第九讲因为笔者水平原因没有找到问题。出现错误还望老师指正。
\begin{figure}[tbph!]
	\centering
	\includegraphics[width=0.7\linewidth]{"figures/第八讲 第11页"}
	\caption[第八讲 第11页]{}
	\label{fig:11页}
\end{figure}

如图\ref{fig:11页},删除线划掉的"假设空间"应为"优化策略",下方删除线的"LASSO"应为"坐标下降法等梯度算法"

\begin{figure}[tbph!]
	\centering
	\includegraphics[width=0.7\linewidth]{figures/贪婪算法部分}
	\caption[第八讲 第16页]{}
	\label{fig:贪婪}
\end{figure}

如图\ref{fig:贪婪}贪婪算法部分,应该把图中三处划红线的"n"改为"m",因为这里是对于样本进行的加和,不是对维度,本页中也已经展示出来样本是最多m个。
%Complete the assignment now
% *** REFERENCES ***
\newpage
\bibliography{bibliography}
\bibliographystyle{naturemag}
\end{document}
%%%%%%%%%%%%%%%%%%%%%%%%%%%%%%%%%%%%%%%%%%%%%%%%%%%%%%%%%%%%%%%%%%
%%%%%%%%%%%%%%%%%%%%%%%%%%%%%%%%%%%%%%%%%%%%%%%%%%%%%%%%%%%%%%%%%%